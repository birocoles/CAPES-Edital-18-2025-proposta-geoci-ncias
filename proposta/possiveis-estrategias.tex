\renewcommand{\chaptername}{Capítulo}
\chapter{Possíveis estratégias para superar os desafios da área}
\renewcommand{\chaptername}{Possíveis estratégias para superar os desafios da área}
\label{cap:possiveis-estrategias}

Uma vez que as diretrizes atuais de avaliação (Capítulo \ref{cap:CAPES-area-avaliacao}) 
são válidas até 2028, quem assumir a coordenação no âmbito do 
Edital 
n$^{\circ}$18/2025\footnote{\href{https://www.gov.br/capes/pt-br/acesso-a-informacao/acoes-e-programas/avaliacao/sobre-a-avaliacao/editais}{\texttt{Edital n$^{\circ}$18/2025 - Indicação e escolha das  coordenações de área}}},
com mandato no período de 2026--2030, deverá promover discussões sobre 
modificações a serem implementadas a partir de 2029 e que serão válidas para o período
de avaliação quadrienal 2029--2032.

É importante reconhecer que o processo de avaliação da pós-graduação na área de Geociências no Brasil 
vem evoluindo de forma considerável ao longo to tempo e que métricas puramente quantitativas de 
comparação e classificação vem perdendo importância frente aos princípios que têm orientado mudanças globais 
de paradigma na avaliação da ciência e da pós-graduação. 
Importantes iniciativas internacionais na avaliação da ciência e da pós-graduação, bem como de programas em geral são:
\begin{itemize}
	\item \href{https://sfdora.org/}{\textbf{DORA -- Declaration on Research Assessment (2012)}:}
	Propõe abandonar o Journal Impact Factor como indicador de qualidade individual.
	Defende avaliação baseada na qualidade intrínseca do trabalho, e não em métricas de periódicos.
	
	\item \href{https://www.leidenmanifesto.org/}{\textbf{Leiden Manifesto for Research Metrics (2015):}}
	Formula 10 princípios para o uso responsável de métricas, destacando contexto, transparência e validação qualitativa.
	Amplamente adotado por universidades e agências europeias.
	
	\item \href{https://www.wcrif.org/guidance/hong-kong-principles}{\textbf{Hong Kong Principles (2019):}}
	Ligados à integridade científica, defendem recompensar práticas abertas, colaborativas e reprodutíveis.
	
	\item \textbf{\href{https://www.oecd.org/en/topics/sub-issues/development-co-operation-evaluation-and-effectiveness/evaluation-criteria.html}{OECD -- Organisation for Economic Co-operation and Development (2019):}}
	Define seis critérios de avaliação -- relevância, coerência, eficácia, eficiência, impacto e sustentabilidade -- e dois 
	princípios orientadores para seu uso. Esses critérios formam um marco normativo destinado a determinar o mérito ou o valor de uma 
	intervenção de desenvolvimento (como políticas, estratégias, programas, projetos ou atividades) e servem de base para a formulação 
	de julgamentos avaliativos.	
	
	\item \href{https://www.unesco.org/en/legal-affairs/recommendation-open-science}{\textbf{UNESCO -- United Nations Educational, Scientific and Cultural Organization (2021):}}
	Propõe avaliar ciência aberta, colaboração e benefícios sociais.
	
	\item \href{https://www.coara.org/}{\textbf{CoARA -- Coalition for Advancing Research Assessment (2022):}}
	Iniciativa europeia para reformular os sistemas de avaliação da pesquisa, enfatizando diversidade, qualidade e impacto social.
	Envolve mais de 700 instituições, inclusive agências de fomento.
\end{itemize}
A partir destas iniciativas, é possível definir alguns princípios estruturantes em torno da avaliação da ciência e da
pós-graduação:
\begin{itemize}
	\item \textbf{Contextualização:} 
	Avaliar conforme a missão e o ambiente institucional.
	
	\item \textbf{Múltiplas dimensões de valor:}
	Considerar não apenas quantidade de produção, mas também relevância, diversidade, impacto social e formação humana.

	\item \textbf{Combinação de métodos:}
	Integrar indicadores quantitativos com avaliação qualitativa por pares.
	
	\item \textbf{Participação e transparência:}
	Envolver os atores avaliados e explicitar critérios e pesos.
	
	\item \textbf{Aprendizagem institucional:}
	Usar a avaliação para aprimorar políticas e práticas, e não apenas classificar.
	
	\item \textbf{Ética e integridade:}
	Garantir que os incentivos avaliativos não deturpem a prática científica.
\end{itemize}

Parte deste princípios podem ser reconhecidos no processo de avaliação atual, no que diz respeito 
(i) ao uso combinado de relatórios narrativos e indicadores quantitativos (princípio da Combinação de Métodos),
(ii) nos esforços para avaliar não apenas o volume de diversos tipos de produção intelectual
(artigos, software, dados, inovação, extensão), mas também a originalidade, relevância e impacto 
de um conjunto de produções intelectuais destacadas pelos próprios programas (princípio das Múltiplas dimensões de valor), 
bem como (iii) na valorização do planejamento estratégico e da autoavaliação dos programas 
(princípio da Aprendizagem Institucional).
Contudo, acredito que algumas estratégias podem ser definidas com base nestes princípios para 
aprimorar o processo avaliativo:
\begin{itemize}
	\item Usar critérios regionalizados, que levem em conta a diversidade dos programas, suas diferentes missões e 
	contextos locais, demográficos e econômicos. Esta ideia foi sugerida no seminário de meio termo de 2023.
	
	\item Repensar os pesos dos critérios atuais para que as ações de colaboração sejam mais vantajosas do que
	os esforços direcionados a produção intelectual interna, sem colaboração, e captação de recursos financeiros 
	a partir de iniciativas individuais de docentes e grupos de pesquisa junto ao setor produtivo não-acadêmico.
	
	\item Propor estratégias que estimulem a colaboração em detrimento da competição entre os PPGs.
	
	\item Melhorar a interlocução entre a coordenação de área e os PPGs com o intuito de (i) esclarecer dúvidas
	referentes ao processo avaliativo, (ii) entender a missão, o ambiente e os objetivos específicos de cada programa, 
	(iii) acompanhar o desenvolvimento anualmente de tal forma que os programas tenham tempo hábil para 
	corrigir rumos ao longo da quadrienal.
	
	\item Valorizar o planejamento estratégico no que diz respeito às ações para (i) corrigir os problemas
	apontados na avaliação quadrienal, (ii) melhorar o desempenho dos programas nas diferentes dimensões da
	avaliação e (iii) promover o enquadramento das pesquisas e atuação do programa de pós-graduação em
	termos dos \href{https://odsbrasil.gov.br/}{Objetivos de Desenvolvimento Sustentável (ODS)}.
	
	\item Em vez de avaliar ``produtos'' destacados pelos programas de forma isolada (artigos, teses, dissertações, etc), 
	avaliar projetos destacados pelos programas no que diz respeito à coerência em relação ao 
	planejamento estratégico, formação de recursos humanos, produção intelectual com participação de discentes e egressos,
	ações de colaboração com programas de níveis diferentes, ações de divulgação científica para o público em geral, 
	envolvimento do corpo docente do próprio programa, entre outros aspectos. 
	É de se esperar que projetos desenvolvidos em programas com nível de maturidade elevado produzam resultados 
	associados a todas as dimensões do processo de avaliação\footnote{\href{https://www.gov.br/capes/pt-br/centrais-de-conteudo/documentos/avaliacao/19052025_20250502_DocumentoReferencial_FICHA.pdf}{ \texttt{Diretrizes comuns da avaliação de permanência dos PPGs, Capítulo 3}}}.
	
	\item Considerar a produção científica em diferentes janelas de tempo, de tal forma que seja possível melhor avaliar
	o impacto destas na sociedade e comunidade científica.
\end{itemize}