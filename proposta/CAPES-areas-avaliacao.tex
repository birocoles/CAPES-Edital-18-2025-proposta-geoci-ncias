\renewcommand{\chaptername}{Capítulo}
\chapter{A CAPES e o papel indutor das Áreas de Avaliação}
\renewcommand{\chaptername}{A CAPES e o papel indutor das Áreas de Avaliação}
\label{cap:CAPES-area-avaliacao}

A CAPES publicou a Portaria 
109/2025\footnote{\href{https://cad.capes.gov.br/ato-administrativo-detalhar?idAtoAdmElastic=17995}{\texttt{Portaria 109/2025 - 25/04/2025 - DOU de 28/04/2025 -- Seção 1 -- p. 55-57}}}, 
que contém normas gerais para o processo de Avaliação de Permanência dos Programas de 
Pós-Graduação stricto sensu.
Também publicou o Documento Referencial\footnote{\href{https://www.gov.br/capes/pt-br/centrais-de-conteudo/documentos/avaliacao/19052025_20250502_DocumentoReferencial_FICHA.pdf}{ \texttt{Diretrizes comuns da avaliação de permanência dos PPGs}}}, 
que apresenta a sistematização da metodologia da Avaliação 
de Permanência, estabelecendo uma sequência lógica de procedimentos e orientações gerais à 
comunidade da pós-graduação e à sociedade, de forma a promover transparência. Representa a 
referência às diretrizes comuns para a Avaliação de Permanência ciclo 2025-2028 e Avaliação 
Quadrienal 2029. 
A partir destas diretrizes comuns, a área de Geociências publicou seu 
documento de área \footnote{\href{https://www.gov.br/capes/pt-br/acesso-a-informacao/acoes-e-programas/avaliacao/sobre-a-avaliacao/areas-avaliacao/sobre-as-areas-de-avaliacao/colegio-de-ciencias-exatas-tecnologicas-e-multidisciplinar/ciencias-exatas-e-da-terra/copy_of_GEOCIENCIAS_DOCAREA_2025_2028.pdf}{\texttt{Documento de área para o período 2025--2028}}}
e ficha de avaliação\footnote{\href{https://www.gov.br/capes/pt-br/acesso-a-informacao/acoes-e-programas/avaliacao/sobre-a-avaliacao/areas-avaliacao/sobre-as-areas-de-avaliacao/colegio-de-ciencias-exatas-tecnologicas-e-multidisciplinar/ciencias-exatas-e-da-terra/copy_of_GEOCIENCIAS_FICHA_2025_2028.pdf}{\texttt{Ficha de avaliação para o período 2025--2028}}}. 
Estes documentos devem nortear a avaliação quadrienal que acontecerá em 2029, referente ao 
período 2025--2028, independente que quem assumir a coordenação da área de geociências.
Nesse sentido, quem assumir a coordenação da área de Geociências conforme o 
Edital 
n$^{\circ}$18/2025\footnote{\href{https://www.gov.br/capes/pt-br/acesso-a-informacao/acoes-e-programas/avaliacao/sobre-a-avaliacao/editais}{\texttt{Edital n$^{\circ}$18/2025 - Indicação e escolha das  coordenações de área}}} 
deverá seguir as mesmas diretrizes de avaliação atuais.

Segundo o Documento Referencial, a Coordenação de Aperfeiçoamento de Pessoal de Nível Superior 
(CAPES) realiza, desde 1976, a avaliação dos programas de pós-graduação (PPGs) stricto sensu 
no Brasil, que englobam os cursos de mestrado e doutorado. Esse processo é essencial para 
promover a qualidade da formação de recursos humanos, estimular a produção de conhecimento e 
impulsionar o desenvolvimento científico e tecnológico do país.
A avaliação da CAPES estabelece as condições mínimas para a criação de novos cursos e assegura 
a qualidade dos programas já existentes. Além disso, permite identificar assimetrias regionais 
e mapear áreas estratégicas para o avanço da pesquisa nacional, orientando a expansão e o 
fortalecimento da pós-graduação em todo o território brasileiro.

O Documento Referencial também cita a Diretoria de Avaliação (DAV) da CAPES como instância 
responsável pela condução do processo avaliativo. Compete a ela 
propor normas e diretrizes gerais aplicáveis a todos os PPGs e executar as ações de avaliação 
previstas em um calendário anual.
Para organizar o processo de avaliação, a CAPES estrutura os campos do conhecimento em três 
Colégios, que reúnem nove Grandes Áreas, subdivididas em 
cinquenta Áreas de Avaliação. Essa estrutura busca assegurar coerência e equidade na aplicação 
dos critérios avaliativos entre programas que compartilham campos de saber relacionados.
As Áreas de Avaliação, por meio de seus coordenadores e em diálogo com a comunidade acadêmica, 
elaboram documentos orientadores contendo as diretrizes 
específicas que norteiam a avaliação dos PPGs.

É importante destacar que a CAPES não define eixos estratégicos de pesquisa de forma 
centralizada. Seu papel é avaliar e fomentar a qualidade global da 
pós-graduação stricto sensu, e não determinar temas ou agendas científicas.
Por outro lado, as Áreas de Avaliação exercem influência indutiva sobre o direcionamento dos 
programas, uma vez que seus critérios e diretrizes específicas acabam orientando as práticas e 
prioridades de cada PPG.