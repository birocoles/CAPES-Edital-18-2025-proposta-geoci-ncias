\renewcommand{\chaptername}{Capítulo}
\chapter{Competição $\times$ Colaboração}
\renewcommand{\chaptername}{Competição $\times$ Colaboração}
\label{cap:competicao-colaboracao}

É razoável argumentar que a estrutura da ficha de avaliação 
atual\footnote{\href{https://www.gov.br/capes/pt-br/acesso-a-informacao/acoes-e-programas/avaliacao/sobre-a-avaliacao/areas-avaliacao/sobre-as-areas-de-avaliacao/colegio-de-ciencias-exatas-tecnologicas-e-multidisciplinar/ciencias-exatas-e-da-terra/copy_of_GEOCIENCIAS_FICHA_2025_2028.pdf}{\texttt{Ficha de avaliação para o período 2025--2028}}} 
encoraja a competição baseada no desempenho interno e na captação de recursos para a maioria 
dos programas (Notas 3, 4 e 5), enquanto a colaboração institucional profunda é um requisito 
de excelência, mas de maior esforço.

\section{Produção intelectual interna}

A avaliação baseia-se nos princípios de comparabilidade e classificação, 
que são mecanismos inerentes à competição. O quesito central é a Formação e Produção 
Intelectual (Quesito 2).
A avaliação prioriza a produção intelectual que envolve a formação, sendo que artigos e 
capítulos de livro com a participação discentes e egressos tem peso igual a 1,0 para o 
cálculo da produção per capita. Este é o caminho mais direto para pontuar no Quesito 2 
(Produção Intelectual), que corresponde a 60\% da nota do quesito tanto para programas 
acadêmicos quanto profissionais.
Artigos e capítulos de livro sem a participação discente e com autoria de somente docentes 
permanentes de mais de um programa tem a pontuação dividida entre os programas envolvidos 
(peso 0,5). Isso força os docentes a garantir que a produção seja resultado da formação dos 
alunos, um processo que ocorre majoritariamente dentro das linhas de pesquisa e projetos 
internos do próprio programa.
Particularmente, considero que, para fins de avaliação dos PPGs, a publicação com docentes 
e egressos deve ser priorizada, uma vez a formação de recursos humanos altamente 
qualificados é a missão principal da pós-graduação.
Por outro lado, também acho que a ênfase na produção intelectual interna é algo que 
inibe a colaboração com outros programas.

\section{Captação de recursos}

O foco na captação de recursos financeiros, seja por meio de agências de fomento públicas ou junto 
ao setor produtivo não-acadêmico (empresas privadas), é um elemento explícito e valorizado na avaliação.
A capacidade do corpo docente permanente em captar recursos financeiros à pesquisa por meio de 
agências públicas ou privadas, nacionais e internacionais, indústrias ou similares, é um fator 
quantitativo na avaliação da qualidade e dinâmica do corpo docente (Item 1.1.2 da ficha de 
avaliação).
Além disso, a Inovação, transferência e compartilhamento de conhecimento (Item 3.2, com peso 
de 35\% no Quesito 3 para programas acadêmicos e profissionais) 
é avaliada por meio de indicadores que incluem 
(i) Teses/Dissertações com metodologias e/ou resultados diretamente empregados no setor 
produtivo não-acadêmico; (ii) Projetos de pesquisa desenvolvidos no quadriênio com o 
setor produtivo não-acadêmico; (iii) A capacidade de captação de recursos financeiros junto ao 
setor produtivo não-acadêmico e a demonstração da contribuição prática (econômica, social e 
ambiental) resultantes da implementação do produto tecnológico para o caso de programas 
profissionais.
Se a captação de recursos do setor produtivo depender primariamente de conexões pré-existentes 
ou esforços individuais de docentes e grupos de pesquisa, esse caminho pode 
ser menos complexo em termos de gestão institucional do que montar um programa associativo 
interinstitucional. Essa priorização, quantificada e incorporada nos Quesitos 1 
(Programa) e 3 (Impacto), atua como um forte indutor de ações de competição por financiamento.

\section{Solidariedade e Nucleação}

Segundo as diretrizes atuais de avaliação\footnote{\href{https://www.gov.br/capes/pt-br/centrais-de-conteudo/documentos/avaliacao/19052025_20250502_DocumentoReferencial_FICHA.pdf}{ \texttt{Diretrizes comuns da avaliação de permanência dos PPGs}}}, 
as principais formas de solidariedade ou nucleação
ocorrem por meio dos Projetos de Cooperação Interinstitucional (PCI); dos grupos de pesquisa e
ambientes colaborativos e das redes de colaboração acadêmica entre instituições para 
qualificação de profissionais de nível superior.
As ações de Solidariedade e Nucleação (como Minter/Dinter ou Formas Associativas) incentivadas 
pela área com o intuito de minimizar as assimetrias regionais são 
complexas, exigem recursos, apoio organizacional e esforço institucional coordenado para 
viabilizar a formação de mestres e doutores fora dos centros consolidados.
O foco em colaboração (Solidariedade e Nucleação) é obrigatório apenas para programas de 
excelência (com Notas 6 e 7), minoria entre os PPGs da área de Geociências.
Segundo consta no documento de área mais recente\footnote{\href{https://www.gov.br/capes/pt-br/acesso-a-informacao/acoes-e-programas/avaliacao/sobre-a-avaliacao/areas-avaliacao/sobre-as-areas-de-avaliacao/colegio-de-ciencias-exatas-tecnologicas-e-multidisciplinar/ciencias-exatas-e-da-terra/copy_of_GEOCIENCIAS_DOCAREA_2025_2028.pdf}{\texttt{Documento de área para o período 2025--2028}}}, apenas 
15 programas (26,3\% do total) foram considerados de excelência (7 com nota 6 e 8 com nota 7) 
na quadrienal 2017--2020, enquanto a maioria foi classificada com notas 3 
(10 programas, 17,5\%) e 4 (24 programas, 42,1\%). 
Apenas 8 programas foram classificados com nota 5 (14\%).
Por outro lado, a colaboração para a redução de assimetrias é relevante para todos os 
programas e aparece no Quesito 3 (Impacto).
Segundo consta no Item 3.2, ``a participação em projetos de cooperação entre programas com 
níveis de consolidação diferentes, voltados para a inovação na pesquisa ou o desenvolvimento 
da pós-graduação em regiões ou sub-regiões geográficas com menos recursos financeiros''.
Portanto, programas com notas 3, 4 e 5 são induzidos, mas não obrigados, a colaborar para melhorar 
a nota no Quesito 3.
Entendo que os incentivos a colaboração deveriam ter um peso maior nos critérios de avaliação, 
de tal forma que esta estratégia para a evolução do programas seja mais vantajosa do que 
a produção intelectual predominantemente interna e a busca por recursos financeiros baseada em
conexões individuais de docentes e/ou grupos de pesquisa junto ao setor produtivo não-acadêmico.

