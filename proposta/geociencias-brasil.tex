\renewcommand{\chaptername}{Capítulo}
\chapter{Geociências no Brasil: Desafios da Pós-Graduação}
\renewcommand{\chaptername}{Geociências no Brasil: Consolidação e Desafios da Pós-Graduação}
\label{cap:geociencias-brasil}

Os documentos de área referentes aos dois últimos períodos de avaliação quadrienal 
disponíveis na página da área de 
Geociências\footnote{\href{https://www.gov.br/capes/pt-br/acesso-a-informacao/acoes-e-programas/avaliacao/sobre-a-avaliacao/areas-avaliacao/sobre-as-areas-de-avaliacao/colegio-de-ciencias-exatas-tecnologicas-e-multidisciplinar/ciencias-exatas-e-da-terra/geociencias}{ \texttt{Página da área de Geociências}}} 
indicam que a situação atual da área no Brasil é marcada por um estágio de elevada 
consolidação e maturidade na pós-graduação, juntamente com desafios persistentes em termos de 
assimetria regional, expansão de subáreas específicas e necessidade de alinhamento com 
demandas sociais e ambientais.

\section{Número de PPGs}

O número total de programas de pós-graduação em Geociências era de 57 em 2023, 
tendo alcançado 58 na última avaliação quadrienal (2017--2020). 
O crescimento da área, de aproximadamente 68\% nos últimos 27 anos (1996--2023), é 
considerado lento e inferior ao crescimento geral do sistema de pós-graduação no país. 
Uma possível explicação para 
este padrão sugerida no documento de área mais recente é a similaridade numérica entre os 
cursos de pós-graduação e de graduação existentes no Brasil. Esta explicação se baseia na 
ideia de que os cursos de graduação na área de Geociências são os principais fornecedores 
de estudantes para os programas de pós-graduação da área e que a similaridade numérica 
impõe um teto a oferta de estudantes. 
Ao considerarmos a pertinência desta explicação, é razoável concluir que o crescimento 
dos programas de pós-graduação requer mecanismos para atrair estudantes provenientes 
de cursos de graduação de áreas afins.

\section{Vínculo predominante com instituições públicas}
	
A área possui uma característica singular de estar estreitamente vinculada a 
instituições de ensino superior de natureza predominantemente pública (federais, estaduais 
e institutos de pesquisa). 
Na atualidade, praticamente todos os programas são vinculados a instituições públicas, o 
que os torna vulneráveis às conjunturas políticas e financeiras que definem a abertura de 
concursos e a reposição de pessoal.
Nesse sentido, a penalização de programas pelo tamanho reduzido de seu corpo docente é um 
ponto de tensão nas métricas de avaliação, ainda que a área permita flexibilizar o número 
mínimo de docentes necessários em função de assimetrias regionais, que incluem o 
desenvolvimento econômico, infraestrutura, qualidade de vida e serviços públicos, entre 
outros.
	
\section{Assimetrias regionais}

Há uma nítida assimetria regional na distribuição de PPGs na área de Geociências. 
A região Sudeste concentra 25 programas (44\%), seguida pelas regiões Nordeste com 13 
programas (23\%), Sul com 10 programas (17\%), Norte com 5 programas (9\%) e região 
Centro-Oeste com 4 programas (7\%). Ou seja, com 16\% dos programas estão localizados nas 
regiões Norte (9\%) e Centro-Oeste (7\%), ainda que estas correspondam a 63\% do 
território nacional. 
Para contornar este problema, a Área de Geociências incentiva a criação de novos programas 
em formas associativas e a participação em projetos de cooperação envolvendo programas com 
diferentes níveis de consolidação. 
É importante ressaltar que há também uma acentuada assimetria regional na distribuição de 
bolsas 
CAPES\footnote{\href{https://www.gov.br/capes/pt-br/acesso-a-informacao/acoes-e-programas/bolsas}{\texttt{Painel dinâmico da CAPES}}}, 
CNPq\footnote{\href{http://bi.cnpq.br/painel/fomento-cti/}{\texttt{Painel dinâmico do CNPq}}} e 
Projetos de P\&D com a indústria de óleo e gás\footnote{\href{https://www.gov.br/anp/pt-br/centrais-de-conteudo/paineis-dinamicos-da-anp/paineis-dinamicos-sobre-exploracao-e-producao-de-petroleo-e-gas}{\texttt{Painel dinâmico da ANP}}}.
Nesse sentido, é necessária especial atenção aos critérios relacionados a 
capacidade do corpo docente permanente em captar recursos financeiros à pesquisa por meio 
de agências públicas ou privadas (nacionais e internacionais), indústrias ou similares, e 
a distribuição desses recursos entre o corpo docente, bem como ao impacto 
que esta captação de recursos causa na sociedade (Itens 3.2 e 3.3 do Quesito 3 ``Impacto'' 
da ficha de 
avaliação\footnote{\href{https://www.gov.br/capes/pt-br/acesso-a-informacao/acoes-e-programas/avaliacao/sobre-a-avaliacao/areas-avaliacao/sobre-as-areas-de-avaliacao/colegio-de-ciencias-exatas-tecnologicas-e-multidisciplinar/ciencias-exatas-e-da-terra/copy_of_GEOCIENCIAS_FICHA_2025_2028.pdf}{\texttt{Ficha de avaliação para o período 2025--2028}}}). 
Se por um lado estes critérios podem 
induzir o caráter inovador da produção intelectual dos PPGs e direcionar esforços na 
solução de questões que afetam a sociedade diretamente, por outro lado eles
podem acentuar as já mencionadas assimetrias regionais e financeiras. 
O setor produtivo não-acadêmico tem significativa autonomia para direcionar seus recursos 
financeiros e o cenário atual sugere que isso não está alinhado aos esforços da área de 
Geociências ou da CAPES como um todo na redução das assimetrias regionais e financeiras.